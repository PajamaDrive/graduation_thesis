\documentclass[a4paper, 10.5pt, twoside]{jreport}

% include
\usepackage{gra_yasuda}
\usepackage{lscape}
\usepackage{graphicx}
\usepackage{here}
\usepackage{color}
\usepackage{amsmath}
\usepackage{subfig}
\usepackage{tascmac}
\usepackage{url}
\usepackage{ascmac}
\usepackage{booktabs}
\usepackage{otf}
\usepackage{comment}



%タイトル
\title{心理的効果を用いた人間とエージェントの繰り返し交渉戦略}
\etitle{Repetitive negotiation strategy of human and agent \\ using psychological effect}

%名前
\author{松下 昌悟}
\eauthor{Shogo MATSUSHITA}

%入学年度
\enteryear{2017}
%卒業年度
\graduateyear{2018}

%学籍番号
\studentnumber{17268508}

%提出日
\date{平成30年1月31日}

\begin{document}

%ここで行ピッチを指定
%フォントを変えるとサイズがリセットされてしまうので注意
\setlength{\baselineskip}{8truemm}


%ここから内容

% Chapter 4
\chapter{予備実験}\label{cha:4}

\section{目的と概要}
提案手法に用いる各種パラメータを決定することを目的としてエージェント同士での交渉を行う実験を行う.
具体的には提案手法の戦略を適用した8つのエージェントと予備実験用の戦略を適用したエージェントでそれぞれ交渉を行う.
本実験では,提案手法の戦略を適用したエージェントの獲得効用が高いほど良いパラメータであると評価する.

\section{実験設定}
本実験では,各論点が0\sim 5の計6つの水準を有する,4つの論点について交渉を行う.

本実験で用いるドメインの詳細およびパラメータを表\ref{tab:pre_domain},表\ref{tab:pre_para}にそれぞれ示す.

\begin{table}[htb]
  \begin{center}
    \caption{予備実験で用いるドメイン}
    \label{tab:pre_domain}
    \begin{tabular}{|c|c|c|c|} \hline
      論点数 & 各論点の水準数 & 繰り返し回数 & BATNA \\ \hline \hline
      4 & 6 & 5 & 4 \\ \hline
    \end{tabular}
  \end{center}
\end{table}

\begin{table}[htb]
  \begin{center}
    \caption{予備実験で用いるパラメータ}
    \label{tab:pre_para}
    \begin{tabular}{|c|c|c|c|c|} \hline
      \alpha の初期値 & \alpha の増分 & \beta の初期値 & \beta の増分 & \beta の更新回数 \\ \hline \hline
      0.0 \sim 8.0 & 0.5 \sim 4.0 & 0.0 \sim 8.0 & 0.5 \sim 4.0 & 1 \sim 10 \\ \hline
    \end{tabular}
  \end{center}
\end{table}

本実験では表\ref{tab:pre_domain}に示したように,1セット5回の交渉を行う.各論点の価値は1セットごとにランダムに変化させる.
各論点の価値は1 \sim 4の間の値で重複はない.
また,提案手法のエージェントにとって価値が一番高いものは予備実験用のエージェントにとって一番価値が低いといったように,一方にとって価値が高いものはもう一方にとって価値が低くなるように設定する.
各論点の価値の例を表\ref{tab:pre_value}に示す.
論点1を例にとると,提案手法のエージェントにとっての価値は4であり,一番価値がある論点となる.
対して予備実験用のエージェントにとっての価値は1であり,一番価値がない論点となる.

\begin{table}[htb]
  \begin{center}
    \caption{各論点の価値}
    \label{tab:pre_value}
    \begin{tabular}{|c|c|c|c|c|} \hline
       & 論点1の価値 & 論点2の価値 & 論点3の価値 & 論点4の価値 \\ \hline \hline
      提案手法のエージェント & 4 & 2 & 3 & 1 \\ \hline
      予備実験用のエージェント & 1 & 3 & 2 & 4 \\ \hline
    \end{tabular}
  \end{center}
\end{table}

\section{実験結果と考察}

\subsection{パラメータ\alpha の実験結果と考察}

\subsection{パラメータ\beta の実験結果と考察}

%内容ここまで

\chapter*{謝辞}
本論文を執筆するにあたり,多数の方々からご指導・ご協力いただきましたことを,心より御礼申し上げます.

指導教員である藤田桂英准教授には,研究の機会を与えていただき,研究の方針に関する助言や発表練習等の
多大なるご指導や助言をいただきましたことを深く感謝いたします.

研究に関する知識のご教示に加えて,本実験の準備を行うにあたってWEBサーバを構築する際にお力添えいただいた松根鷹生様に深く感謝申し上げます.
また,藤田桂英研究室の皆様には研究に必要な知識や意見等をいただいたことを心より感謝いたします.

本実験を行うにあたってお忙しい中ご協力いただいた同期の編入生の方々,および安井貴規様がいなければ本論文は完成に至りませんでした.
心より御礼申し上げます.

最後に,様々な面で私を支えていただいた家族に,心より感謝いたします.ありがとうございました.

\bibliographystyle{plain}
\bibliography{reference}


\begin{comment}
%付録で発表論文をつけてアピールだ!!

\renewcommand{\bibname}{付録 発表論文一覧}
%\chapter{発表論文一覧}

\begin{thebibliography}{99}
\item S. Kakimoto and K. Fujita. 二者間複数論点交渉問題におけるパレートフロント推定手法の提案. Joint Agent Workshop and Symposium, 2014.
\item S. Kakimoto and K. Fujita. Estimating Pareto Fronts using Interdependency between Issues for Bilateral Multi-issue Closed Nonlinear Negotiations. Applications Knowledge and Service Technology for Life, Environment, and Sustainability workshop(KASTLES),2014.
\item S. Kakimoto and K. Fujita. 二者間非線形交渉問題におけるパレートフロント推定を利用した自動交渉エージェントの設計と評価. 情報処理学会 第177回 知能システム研究会, 2014.

\end{thebibliography}

\end{comment}

\end{document}

