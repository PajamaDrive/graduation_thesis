
交渉スキルは友人への頼みごと等の小規模な問題だけでなく企業の提携,国家間の取引等の大規模な問題を解決する際に必要である.
しかし,交渉スキルを高めるには専門家による指導を受け,実践する必要があるため習得コストが非常に高い.
特に,交渉はビジネス,衝突解消,AIなど複数の分野で研究されているが,近年では交渉スキルの教育ツールなどへ応用が可能であるため,人間とエージェントとの交渉への関心が高まっている.

人間とエージェントの交渉では感情表現が及ぼす効果や,単一の交渉において効果のある戦略等が研究されている.
一方で,繰り返し行われる交渉に対応できる戦略が少ないのが現状である.

そこで本論文では,同じ対戦相手と繰り返し行われる交渉に対する戦略として,人間の心理的効果を利用した交渉術である段階的要請法と譲歩的要請法を組み合わせた手法を提案する.
提案手法では,1回の交渉内では時間が経過するごとに提案を受諾する水準を変化させ,同時に相手との交渉回数が増加するごとに水準を変化させる.
2つの水準の変化に対して段階的要請法,譲歩的要請法,水準を変化させない方法の3種類からそれぞれ1つずつ適用し,これらを組み合わせた8種類の戦略を提案し,IAGOのサンプルエージェントと比較を行う.
評価実験を行う前に予備実験を実施し,8種類の戦略を適用したエージェントと予備実験用のエージェントでそれぞれ交渉を行い,段階的要請法と譲歩的要請法に用いるパラメータを決定した.

さらに,決定したパラメータを用いて9人の被験者に対してエージェントと交渉する実験を行なった.
評価実験において,エージェントと人間が得られた効用の差を比較すると8種類の戦略すべてがIAGOのサンプルエージェントより高いことを示した.
