\documentclass[a4paper, 10.5pt, twoside]{jreport}

% include
\usepackage{gra_yasuda}
\usepackage{lscape}
\usepackage{graphicx}
\usepackage{here}
\usepackage{color}
\usepackage{amsmath}
\usepackage{subfig}
\usepackage{tascmac}
\usepackage{url}
\usepackage{ascmac}
\usepackage{booktabs}
\usepackage{otf}
\usepackage{comment}



%タイトル
\title{心理的効果を用いた人間とエージェントの繰り返し交渉戦略}
\etitle{Repetitive negotiation strategy of human and agent \\ using psychological effect}

%名前
\author{松下 昌悟}
\eauthor{Shogo MATSUSHITA}

%入学年度
\enteryear{2017}
%卒業年度
\graduateyear{2018}

%学籍番号
\studentnumber{17268508}

%提出日
\date{平成30年1月xx日}

\begin{document}

%ここで行ピッチを指定
%フォントを変えるとサイズがリセットされてしまうので注意
\setlength{\baselineskip}{8truemm}

%ここから内容

% Chapter 1
\chapter{問題設定}\label{cha:1}

\section{複数論点交渉問題}
本研究では,交渉問題の中でも論点が複数存在する複数論点交渉問題を扱う.エージェント$A_1$と$A_2$が交渉を行う場合を考える.エージェント$a \in \{ A_1, A_2 \}$の目的関数$f$は,$a$の効用関数$U_{a}$と全ての合意案候補集合$S$を用いると式\ref{eq:objectFunc}と表すことができる.
\begin{equation}
  f = \argmax_{s \in S} U_{a}(s)
  \label{eq:objectFunc}
\end{equation}

複数論点交渉問題の場合,目的関数$g$は式\ref{eq:socialSurplus}で表され,この目的関数の値は社会的余剰と呼ばれる.
\begin{equation}
  g = \argmax_{s \in S} \sum_{a \in \{ A_1, A_2 \}} U_{a}(s)
  \label{eq:socialSurplus}
\end{equation}

一つの交渉問題はそれぞれドメインと呼ばれ,論点数$N$のドメインは固有の論点集合$I = \{ i_1, i_2, ..., i_N \}$を持つ.
また,論点$i_k \in I$は選択肢集合$V_k = \{ v_{k1}, v_{k2}, ... ,v_{kn_k} \}$を持つ.ただし,論点$i_k$の選択肢数を$n_k$と定義する.

各論点$i_k \in I$についてそれぞれ選択肢$v_k \in V_k$を一つずつ選んだものを合意案候補(Bid)と呼び,
$s = (v_1, v_2, ..., v_N)$ として表現される.また,全ての合意案候補集合$S$は式\ref{eq:Bid}と表すことができる.
\begin{equation}
  S = \{ s = (v_1, v_2, ..., v_N) \, | \, v_k \in V_k , 1 \leqq k \geqq N \}
  \label{eq:Bid}
\end{equation}

エージェントは各論点$i_k$について重み$w_k$$(\sum_{k = 1}^N w_k = 1)$および選択肢の評価値$eval(v_k \in V_k)$を持つ.
ただし,$eval(v_k \in V_k)$は最大値が1となるように正規化されているものとする.このとき,エージェントの効用関数$U$は式\ref{eq:Utility}となる.
\begin{equation}
  U(s) = \sum_{k = 1}^N w_k \cdot eval(v_k)
  \label{eq:Utility}
\end{equation}
また,各エージェントに対し留保価格(reservation value) 設定される場合がある.
留保価格は合意形成に失敗した際にエージェントが獲得できる効用値である.

\section{繰り返し交渉問題}
本研究では,繰り返し交渉問題を扱う.繰り返し交渉問題は,交渉問題の中でエージェント$A_1$と$A_2$が複数回交渉を行うものを指す.
本研究では,繰り返し交渉問題の中でも特に,以下の2つの関係を満たす交渉問題について取り扱う.
\begin{enumerate}
  \item 各交渉において各論点における各選択肢の一番高い値をエージェント$A_1$が選択した場合の効用$U_1$と$A_2$が選択した場合の効用$U_2$は等しい
  \item $1,2, \cdots n$回目の交渉において$U_1$および$U_2$の値は不変である
\end{enumerate}
エージェント$a = \{ A_1, A_2 \}$にとっての論点$i_k$についての選択肢の実際の価値を$val(v_{ak} \in V_{ak})$とすると,
各交渉におけるエージェント$A_1$と$A_2$の効用値の関係は式\ref{eq:bothUtility}となる.
\begin{equation}
  \sum_{k = 1}^N w_{A_1k} \cdot val(v_{A_1k}) = \sum_{k = 1}^N w_{A_2k} \cdot val(v_{A_2k})
  \label{eq:bothUtility}
\end{equation}

また,$n$回目の交渉におけるエージェント$a = \{ A_1, A_2 \}$にとっての論点$i_k$についての重みを$w_{A_1k}^n$選択肢の実際の価値を$val(v_{ak}^n \in V_{ak}^n)$とすると,$1,2, \cdots ,n$回目の交渉における効用値の関係は式\ref{eq:repeatedUtility}となる.
\begin{equation}
  \sum_{k = 1}^N w_{ak}^1 \cdot val(v_{ak}^1) = \sum_{k = 1}^N w_{ak}^2 \cdot val(v_{ak}^2) = \cdots = \sum_{k = 1}^N w_{ak}^n \cdot val(v_{ak}^n)
  \label{eq:repeatedUtility}
\end{equation}


%内容ここまで

\chapter*{謝辞}
本論文を執筆するにあたり,多数の方々からご指導・ご協力いただきましたことを,心より御礼申し上げます.

指導教員である藤田桂英准教授には,研究の機会を与えていただき,研究の方針に関する助言や発表練習等の
多大なるご指導や助言をいただきましたことを深く感謝いたします.

研究に関する知識のご教示に加えて,本実験の準備を行うにあたってWEBサーバを構築する際にお力添えいただいた松根鷹生様に深く感謝申し上げます.
また,藤田桂英研究室の皆様には研究に必要な知識や意見等をいただいたことを心より感謝いたします.

本実験を行うにあたってお忙しい中ご協力いただいた同期の編入生の方々,および安井貴規様がいなければ本論文は完成に至りませんでした.
心より御礼申し上げます.

最後に,様々な面で私を支えていただいた家族に,心より感謝いたします.ありがとうございました.

\bibliographystyle{plain}
\bibliography{reference}


\begin{comment}
%付録で発表論文をつけてアピールだ!!

\renewcommand{\bibname}{付録 発表論文一覧}
%\chapter{発表論文一覧}

\begin{thebibliography}{99}
\item S. Kakimoto and K. Fujita. 二者間複数論点交渉問題におけるパレートフロント推定手法の提案. Joint Agent Workshop and Symposium, 2014.
\item S. Kakimoto and K. Fujita. Estimating Pareto Fronts using Interdependency between Issues for Bilateral Multi-issue Closed Nonlinear Negotiations. Applications Knowledge and Service Technology for Life, Environment, and Sustainability workshop(KASTLES),2014.
\item S. Kakimoto and K. Fujita. 二者間非線形交渉問題におけるパレートフロント推定を利用した自動交渉エージェントの設計と評価. 情報処理学会 第177回 知能システム研究会, 2014.

\end{thebibliography}

\end{comment}

\end{document}

