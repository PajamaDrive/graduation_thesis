\documentclass[a4paper, 10.5pt, twoside]{jreport}

% include
\usepackage{gra_yasuda}
\usepackage{lscape}
\usepackage{graphicx}
\usepackage{here}
\usepackage{color}
\usepackage{amsmath}
\usepackage{subfig}
\usepackage{tascmac}
\usepackage{url}
\usepackage{ascmac}
\usepackage{booktabs}
\usepackage{otf}
\usepackage{comment}



%タイトル
\title{心理的効果を用いた人間とエージェントの繰り返し交渉戦略}
\etitle{Repetitive negotiation strategy of human and agent \\ using psychological effect}

%名前
\author{松下 昌悟}
\eauthor{Shogo MATSUSHITA}

%入学年度
\enteryear{2017}
%卒業年度
\graduateyear{2018}

%学籍番号
\studentnumber{17268508}

%提出日
\date{平成30年1月31日}

\begin{document}

%ここで行ピッチを指定
%フォントを変えるとサイズがリセットされてしまうので注意
\setlength{\baselineskip}{8truemm}


%ここから内容

% Chapter 1
\chapter{はじめに}\label{cha:1}

\section{背景}
交渉スキルは友人への頼みごと等の日常的な小規模な問題だけでなく企業の提携,国家間の取引等の大規模な問題を解決する際に必要である.
特に,組織的な団体に属する場合は目標を達成するために交渉を行う必要がある場面が多い.
しかし,交渉スキルなどの対人能力は大学を卒業した学生であっても不十分であり\cite{graduate},交渉スキルの不足により不利益を被ることも多い.
交渉スキルを高めるには専門家による指導を受け,体験学習で実践を行う必要があり,受講の費用も高額であるため習得コストが非常に高い.
このように交渉スキルを高めるためのコストは非常に高いが,教育ツールとして交渉エージェントを包含したソフトウェアを用いることで習得にかかるコストを劇的に削減することができる.また,専門家による指導は指導者と受講者の時間的制約もあるが,ソフトウェアによる指導の場合は時間的制約も削減することが可能である.
人間とエージェントの交渉において交渉における5つの原則\cite{nego_principle}のうち3つをどの程度達成しているかを数値化し視覚化する研究\cite{visualize}も行われており,教育ツールへの応用が期待されている.
交渉はビジネス,衝突解消,AIなど複数の分野で研究されているが,前述のように交渉スキルを教育するためのツールなどへ応用が可能であるため,人間とエージェントとの交渉への関心が高まっている\cite{vr}.

近年,行動科学の研究では交渉中に感情が与える影響に関心が高まってきている\cite{behavior}.
感情は重要な社会的機能を有しており,個人の信条,欲望,意図などの情報を伝達する.
例として多くの研究では交渉において怒りは相手からより多くの譲歩を引き出す一方で,喜びは相手からあまり譲歩を引き出すことができないという結果が示されている\cite{angryhappy}.
したがって,交渉相手が怒った場合は,合意に達するために要求を下げ,一方で,交渉相手が喜んだ場合は,戦略的に多くの要求をすることができる.
これらの心理的効果が人間同士の交渉だけではなく人間とエージェントの交渉でも同様な効果があることが示された\cite{emotion}.
このように,人間とエージェントが交渉を行う際も感情などの心理的効果が交渉結果に影響を与えるが,人間とエージェントの交渉ではこれらの影響を考慮していないモデルが多かった.
その要因として,エージェント同士の交渉に用いるエージェントを作成するプラットフォームとしてGenius\cite{genius}などがある一方で,人間と交渉可能なエージェントを作成するために最適なプラットフォームがないという問題点があった.
複数論点交渉問題について扱うColored Trails\cite{ct},Colored TrailsのWEB版であるWebCT\cite{webct},自然言語による交渉に焦点を当てたNegoChat\cite{negochat}などが存在しているが,これらのプラットフォームは単一のチャネルを用いたコミュニケーションに重点を置いており,感情に関する情報を表出するチャネルが含まれていない.
人間と交渉を行うことができるエージェントを作成するためのプラットフォームであるIAGO\cite{iago}は感情やメッセージの送信を行うためのチャネルがあらかじめ用意されており,IAGOが登場してからは,心理的効果を反映したエージェントに関する研究が増えつつある.一方で,繰り返し交渉に対応できる戦略が少ないのが現状である.


また,作成したエージェントを用いて交渉を行い,個人効用や社会的余剰の値を競い合う自動交渉エージェント競技会(Automated Negotiation Agent Competition(ANAC))が2010年から毎年開催されている\cite{anac2010-2015,anac2016,anac2017,anac2018}.
ANACでは2016年まではエージェント同士の交渉を行うリーグのみが開催されていたが,自動交渉はDiplomacyなどの交渉を行うゲームAIや人間との交渉に応用されることが期待されており,ANACでは2017年からDiplomacyを取り扱うDiplomacy Strategy Game League,人間とエージェントが交渉を行うHuman-Agent Negotiation Leagueが開催された.Human-Agent Negotiation Leagueではエージェントを作成するためのプラットフォームとしてIAGOが採用されており,2018年からは繰り返し交渉を取り扱っており,繰り返し交渉に対応した戦略に対する関心が高まってきている.

\section{本研究の目的}
本研究では,人間と交渉を行うことが可能でかつ同じ相手と繰り返し交渉する際に個人効用が高くなるようなエージェントを提案することを目的とする.
具体的には,交渉時間の増加,交渉回数の増加に応じて相手の提案を受諾する水準を変化させることで繰り返し交渉に対応する.
本研究では,提案を受諾する水準を変化させる手法として人間の心理的効果を利用した交渉術である段階的要請法および譲歩的要請法に着目し,
これらの手法を組み合わせて人間とエージェントの自動交渉に用いることで,繰り返し交渉に対応可能な戦略を提案する.
また,提案手法およびIAGOのベースラインのエージェントと人間が交渉を行う実験を実施し評価を行うことで,提案手法の有効性を示すことを目的とする.

\section{本論文の構成}
以下に本論文の構成を述べる.第2章では,関連研究として,自動交渉エージェント競技会の概要,IAGOの概要,心理的効果が交渉に与える影響についての研究について述べる.
第3章では,本研究で取り扱う交渉問題である複数論点交渉問題と繰り返し交渉問題について述べる.
第4章では,提案手法の詳細として,段階的要請法および譲歩的要請法とこれらを組み合わせた繰り返し戦略について述べる.
第5章では,繰り返し戦略に用いるパラメータを決定するために行なった予備実験の結果について述べる.
第6章では,予備実験で決定したパラメータを用いて被験者に対して行なった評価実験の結果について述べる.
最後に,第7章では本研究のまとめと今後の課題を示す.

%内容ここまで

\chapter*{謝辞}
本論文を執筆するにあたり,多数の方々からご指導・ご協力いただきましたことを,心より御礼申し上げます.

指導教員である藤田桂英准教授には,研究の機会を与えていただき,研究の方針に関する助言や発表練習等の
多大なるご指導や助言をいただきましたことを深く感謝いたします.

研究に関する知識のご教示に加えて,本実験の準備を行うにあたってWEBサーバを構築する際にお力添えいただいた松根鷹生様に深く感謝申し上げます.
また,藤田桂英研究室の皆様には研究に必要な知識や意見等をいただいたことを心より感謝いたします.

本実験を行うにあたってお忙しい中ご協力いただいた同期の編入生の方々,および安井貴規様がいなければ本論文は完成に至りませんでした.
心より御礼申し上げます.

最後に,様々な面で私を支えていただいた家族に,心より感謝いたします.ありがとうございました.

\bibliographystyle{plain}
\bibliography{reference}


\begin{comment}
%付録で発表論文をつけてアピールだ!!

\renewcommand{\bibname}{付録 発表論文一覧}
%\chapter{発表論文一覧}

\begin{thebibliography}{99}
\item S. Kakimoto and K. Fujita. 二者間複数論点交渉問題におけるパレートフロント推定手法の提案. Joint Agent Workshop and Symposium, 2014.
\item S. Kakimoto and K. Fujita. Estimating Pareto Fronts using Interdependency between Issues for Bilateral Multi-issue Closed Nonlinear Negotiations. Applications Knowledge and Service Technology for Life, Environment, and Sustainability workshop(KASTLES),2014.
\item S. Kakimoto and K. Fujita. 二者間非線形交渉問題におけるパレートフロント推定を利用した自動交渉エージェントの設計と評価. 情報処理学会 第177回 知能システム研究会, 2014.

\end{thebibliography}

\end{comment}

\end{document}

