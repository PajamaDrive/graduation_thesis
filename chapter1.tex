\documentclass[a4paper, 10.5pt, twoside]{jreport}

% include
\usepackage{gra_yasuda}
\usepackage{lscape}
\usepackage{graphicx}
\usepackage{here}
\usepackage{color}
\usepackage{amsmath}
\usepackage{subfig}
\usepackage{tascmac}
\usepackage{url}
\usepackage{ascmac}
\usepackage{booktabs}
\usepackage{otf}
\usepackage{comment}



%タイトル
\title{心理的効果を用いた人間とエージェントの繰り返し交渉戦略}
\etitle{Repetitive negotiation strategy of human and agent \\ using psychological effect}

%名前
\author{松下 昌悟}
\eauthor{Shogo MATSUSHITA}

%入学年度
\enteryear{2017}
%卒業年度
\graduateyear{2018}

%学籍番号
\studentnumber{17268508}

%提出日
\date{平成30年1月31日}

\begin{document}

%ここで行ピッチを指定
%フォントを変えるとサイズがリセットされてしまうので注意
\setlength{\baselineskip}{8truemm}


%ここから内容

% Chapter 1
\chapter{はじめに}\label{cha:1}

\section{背景}
交渉スキルは友人への頼みごと等の小規模な問題だけでなく企業の提携,国家間の取引等の大規模な問題を解決する際に必要である.
しかし,交渉スキルを高めるには専門家による指導を受け,実践する必要があるため習得コストが非常に高い.
交渉はビジネス,衝突解消,AIなど複数の分野で研究されているが,近年では交渉スキルの教育ツールなどへ応用が可能であるため,人間とエージェントとの交渉への関心が高まっている.

人間とエージェントとの交渉では感情表現が及ぼす効果や,単一の交渉において効果のある戦略等が研究されている.
一方で,繰り返し行われる交渉に対応できる戦略が少ないのが現状である.

\section{本研究の目的}
本研究では,人間の心理的効果を利用した交渉術である段階的要請法および譲歩的要請法に着目し,これらの手法を人間とエージェントの自動交渉に用いることで,
繰り返し交渉で有効な戦略を提案する.

また,提案手法およびIAGOのベースラインのエージェントと人間による交渉を行う実験を実施し,提案手法の有効性を評価することを目的とする.

\section{本論文の構成}
以下に本論文の構成を述べる.第2章では,関連研究として,自動交渉エージェント競技会の概要とIAGOの概要について述べる.
第3章では,提案手法の詳細として,段階的要請法および譲歩的要請法とこれらを組み合わせた繰り返し戦略について述べる.
第4章では繰り返し戦略に用いるパラメータを決定する予備実験の結果について述べる.
その後,第5章では評価実験についての結果と考察を示す.
最後に,第6章では本研究のまとめと今後の課題を示す.

%内容ここまで

\chapter*{謝辞}
本論文を執筆するにあたり,多数の方々からご指導・ご協力いただきましたことを,心より御礼申し上げます.

指導教員である藤田桂英准教授には,研究の機会を与えていただき,研究の方針に関する助言や発表練習等の
多大なるご指導や助言をいただきましたことを深く感謝いたします.

研究に関する知識のご教示に加えて,本実験の準備を行うにあたってWEBサーバを構築する際にお力添えいただいた松根鷹生様に深く感謝申し上げます.
また,藤田桂英研究室の皆様には研究に必要な知識や意見等をいただいたことを心より感謝いたします.

本実験を行うにあたってお忙しい中ご協力いただいた同期の編入生の方々,および安井貴規様がいなければ本論文は完成に至りませんでした.
心より御礼申し上げます.

最後に,様々な面で私を支えていただいた家族に,心より感謝いたします.ありがとうございました.

\bibliographystyle{plain}
\bibliography{reference}


\begin{comment}
%付録で発表論文をつけてアピールだ!!

\renewcommand{\bibname}{付録 発表論文一覧}
%\chapter{発表論文一覧}

\begin{thebibliography}{99}
\item S. Kakimoto and K. Fujita. 二者間複数論点交渉問題におけるパレートフロント推定手法の提案. Joint Agent Workshop and Symposium, 2014.
\item S. Kakimoto and K. Fujita. Estimating Pareto Fronts using Interdependency between Issues for Bilateral Multi-issue Closed Nonlinear Negotiations. Applications Knowledge and Service Technology for Life, Environment, and Sustainability workshop(KASTLES),2014.
\item S. Kakimoto and K. Fujita. 二者間非線形交渉問題におけるパレートフロント推定を利用した自動交渉エージェントの設計と評価. 情報処理学会 第177回 知能システム研究会, 2014.

\end{thebibliography}

\end{comment}

\end{document}

