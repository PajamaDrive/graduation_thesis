\documentclass[a4paper, 10.5pt, twoside]{jreport}

% include
\usepackage{gra_yasuda}
\usepackage{lscape}
\usepackage{graphicx}
\usepackage{here}
\usepackage{color}
\usepackage{amsmath}
\usepackage{subfig}
\usepackage{tascmac}
\usepackage{url}
\usepackage{ascmac}
\usepackage{booktabs}
\usepackage{otf}
\usepackage{comment}



%タイトル
\title{心理的効果を用いた人間とエージェントの繰り返し交渉戦略}
\etitle{Repetitive negotiation strategy of human and agent \\ using psychological effect}

%名前
\author{松下 昌悟}
\eauthor{Shogo MATSUSHITA}

%入学年度
\enteryear{2017}
%卒業年度
\graduateyear{2018}

%学籍番号
\studentnumber{17268508}

%提出日
\date{平成30年1月31日}

\begin{document}

%ここで行ピッチを指定
%フォントを変えるとサイズがリセットされてしまうので注意
\setlength{\baselineskip}{8truemm}


%ここから内容

% Chapter 6
\chapter{おわりに}\label{cha:6}
\section{まとめ}
本研究では,同じ相手と繰り返し交渉する場合に適応可能な戦略として,段階的要請法,譲歩的要請法を組み合わせて受諾水準を変化させる戦略を提案した.
提案した戦略をIAGOのPinocchioエージェントに適応させたエージェントと対戦させ,エージェントのパラメータを決定するために予備実験を行なった.

予備実験を行なった結果,Foot戦略において初期値および増分の値を高く設定すると個人効用が低くなり,Door戦略において初期値を高く,増分を低く設定すると個人効用が高くなる傾向があることが確認できた.
また,予備実験で決定したパラメータを設定したエージェントを用いて被験者に対して評価実験を行なった.
評価実験を行なった結果,提案した戦略すべてが繰り返し交渉に対応していないNotNotより個人効用の差が大きくなったが,
社会的余剰はNotNotよりも低い戦略が多く,良好な関係を構築できていないことが確認できた.

\section{今後の課題}
\begin{description}
  \item[人間との関係構築に関する課題]~\\
  今回の戦略では受諾水準を変化させることで個人効用を上昇させたが,交渉が決裂するよりは公平でなくても提案を受諾した方が良いため仕方なく合意に至ったというケースが多く,両者が歩み寄るような良好な関係を構築できたとは言い難い.関係構築がうまくいかない場合,同じ相手と交渉をするときに不利益を被ってしまうため,良好な関係を保ちつつ個人効用を上昇させるような繰り返し交渉戦略を提案する必要がある.
  \item[パラメータの値に関する課題]~\\
  予備実験ではエージェント同士で交渉を行うことで評価実験に用いるパラメータを決定した.しかし,評価実験では人間とエージェントが交渉を行うため,パラメータの値が最適な値でない可能性がある.そのため,パラメータの調整も含めて人間とエージェントで交渉を行うことでより良い結果になる可能性がある.
  \item[受諾水準の変化に関する課題]~\\
  本稿では受諾水準を時間経過,交渉回数の2種類で変化させた.時間経過による変化はステップ関数的に,交渉回数による変化は線形関数的に変化させたが,これらを対数関数など他の関数を用いて変化させることで人間の譲歩をより再現することができ,社会的余剰を高めつつ個人効用を高めることができる可能性がある.
  \item[被験者数による実験結果の誤差に関する課題]~\\
  評価実験では9人の被験者に対して実験を行なった.エージェントと交渉する順番による順序効果を相殺するために$9n$人に対して実験を行う必要があり,本稿では$n = 1$として実験を行なった.しかし,$n = 1$では各個人の誤差が結果に反映されてしまい,結果が正しくない可能性がある.したがって,$n$の値を大きくして実験を行う必要がある.
\end{description}


%内容ここまで

\chapter*{謝辞}
本論文を執筆するにあたり,多数の方々からご指導・ご協力いただきましたことを,心より御礼申し上げます.

指導教員である藤田桂英准教授には,研究の機会を与えていただき,研究の方針に関する助言や発表練習等の
多大なるご指導や助言をいただきましたことを深く感謝いたします.

研究に関する知識のご教示に加えて,本実験の準備を行うにあたってWEBサーバを構築する際にお力添えいただいた松根鷹生様に深く感謝申し上げます.
また,藤田桂英研究室の皆様には研究に必要な知識や意見等をいただいたことを心より感謝いたします.

本実験を行うにあたってお忙しい中ご協力いただいた同期の編入生の方々,および安井貴規様がいなければ本論文は完成に至りませんでした.
心より御礼申し上げます.

最後に,様々な面で私を支えていただいた家族に,心より感謝いたします.ありがとうございました.

\bibliographystyle{plain}
\bibliography{reference}


\begin{comment}
%付録で発表論文をつけてアピールだ!!

\renewcommand{\bibname}{付録 発表論文一覧}
%\chapter{発表論文一覧}

\begin{thebibliography}{99}
\item S. Kakimoto and K. Fujita. 二者間複数論点交渉問題におけるパレートフロント推定手法の提案. Joint Agent Workshop and Symposium, 2014.
\item S. Kakimoto and K. Fujita. Estimating Pareto Fronts using Interdependency between Issues for Bilateral Multi-issue Closed Nonlinear Negotiations. Applications Knowledge and Service Technology for Life, Environment, and Sustainability workshop(KASTLES),2014.
\item S. Kakimoto and K. Fujita. 二者間非線形交渉問題におけるパレートフロント推定を利用した自動交渉エージェントの設計と評価. 情報処理学会 第177回 知能システム研究会, 2014.

\end{thebibliography}

\end{comment}

\end{document}

